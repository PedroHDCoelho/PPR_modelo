\chapter{Descrição dos Instrumentos de Dosimetria}
\label{ch:descricao_instrumentos-dosimetria}
A Instalação dispõe de (2) dois Conjuntos Dosimétricos, como listado na tabela \ref{tab:conjuntos_dosimetricos}, para realização de dosimetrias absolutas e os respectivos Certificados de Calibração mais recentes dos conjuntos dosimétricos e demais instrumentos de de medição, listados na tabela \ref{tab:instrumentos_medicao} estão dispostos nas seções \ref{sec:certificados_conjuntos-dosimetricos} e \ref{sec:certificados_demais}, nesse anexo.	:

\section{Câmaras de Ionização e Eletrômetros}
\label{sec:camaras_eletrometros}

\begin{table}[!h]
    \centering
    \resizebox{\linewidth}{!}{
    \scriptsize
    \begin{NiceTabular}{|p{5cm}|p{2.5cm}|p{2cm}|p{4cm}|}\hline%
    \rowcolor{\colorTABLE}
        \color{white}\bfseries Item & \color{white}\bfseries Fabricante & \color{white}\bfseries Modelo  & \color{white}\bfseries Nº Série\\\hline\hline
        \csvreader[head to column names, late after line = \\\hline]{./anexos/CSVs/conjunto_dosimetrico.csv}{}{%
        \obj & \fabricante & \modelo & \nserie }%
        \hline
    \end{NiceTabular}
    }
    \caption{Conjuntos dosimétricos.}
    \label{tab:conjuntos_dosimetricos}
\end{table}

  
\section{Termômetro(s), Barômetro(s), Nível(is), Régua(s) e Cronômetro(s)}
\label{sec:ter_bar_niv_reg_cro}
A Instalação dispõe de (1) um Termômetro, (1) um Barômetro, (1) um Nível Digital e (1) uma Régua Calibrados, além de (1) Cronômetro, listados na tabela \ref{tab:instrumentos_medicao}, e os respectivos Certificados de Calibração mais recentes dos Instrumentos a seguir encontram-se disponíveis no final desse anexo, na seção \ref{sec:certificados_demais}.

\begin{table}[!h]
    \centering
    \resizebox{\linewidth}{!}{
    \scriptsize
    \begin{NiceTabular}{|p{5cm}|p{2.5cm}|p{2cm}|p{2cm}|}\hline%
    \rowcolor{\colorTABLE}
        \color{white}\bfseries Item & \color{white}\bfseries Fabricante & \color{white}\bfseries Modelo  & \color{white}\bfseries Nº Série\\\hline\hline
        \csvreader[head to column names, late after line = \\\hline]{./anexos/CSVs/instrumentos_medicao.csv}{}{%
        \obj & \fabricante & \modelo & \nserie }%
        \hline
    \end{NiceTabular}
    }
    \caption{Outros instrumentos de medição.}
    \label{tab:instrumentos_medicao}
\end{table}

\pagebreak
\section{Fantomas}
\label{sec:fantomas}
A Instalação dispõe de (1) um fantoma de água para dosimetria absoluta, como listado na tabela \ref{tab:fantoma}:

\begin{table}[!h]
    \centering
    \resizebox{\linewidth}{!}{
    \scriptsize
    \begin{NiceTabular}{|p{5cm}|p{2.5cm}|p{2cm}|p{2.5cm}|p{2.5cm}|}\hline%
    \rowcolor{\colorTABLE}
        \color{white}\bfseries Item & \color{white}\bfseries Fabricante & \color{white}\bfseries Modelo  & \color{white}\bfseries Dimensões & \color{white}\bfseries Material\\\hline\hline
        \csvreader[head to column names, late after line = \\\hline]{./anexos/CSVs/fantoma.csv}{}{%
        \obj & \fabricante & \modelo & \dimensoes & \material }%
        \hline
    \end{NiceTabular}
    }
    \caption{Fantomas.}
    \label{tab:fantoma}
\end{table}
    
\section{Demais Equipamentos}
\label{sec:outros_eqp}
A Instalação dispõe de equipamentos específicos para realização de Controle de Qualidade, conforme listado na tabela \ref{tab:outros_detectores}:

\begin{table}[!h]
    \centering
    \resizebox{\linewidth}{!}{
    \scriptsize
    \begin{NiceTabular}{|p{3cm}|p{2.5cm}|p{2.5cm}|p{2cm}|p{2cm}|p{1cm}|}\hline%
    \rowcolor{\colorTABLE}
        \color{white}\bfseries Item & \color{white}\bfseries Fabricante & \color{white}\bfseries Modelo & \color{white}\bfseries Nº Série  & \color{white}\bfseries Data Fabricação & \color{white}\bfseries Tipo\\\hline\hline
        \csvreader[head to column names, late after line = \\\hline]{./anexos/CSVs/outros_detectores.csv}{}{%
        \obj & \fabricante & \modelo & \nserie & \data & \tipo }%
        \hline
    \end{NiceTabular}
    }
    \caption{Fantoma.}
    \label{tab:outros_detectores}
\end{table}

    
\section{Descrição dos Controles de Qualidade dos Equipamentos de Dosimetria}
Os equipamentos de dosimetria passam por procedimentos de aferições de desempenho trimestrais, contemplando testes de fuga do conjunto câmara-eletrômetro, medidas de reprodutibilidade e linearidade. Ademais, cada conjunto passa, conforme normas vigentes, por calibração em laboratório secundário bianualmente, porém nunca ambos ao mesmo tempo.
Os equipamentos descritos nas seções \ref{sec:camaras_eletrometros} e \ref{sec:ter_bar_niv_reg_cro}  passam por calibração em laboratório de metrologia anualmente, conforme normas vigentes.

\pagebreak
\section{Certificados de Calibração dos Conjuntos Dosimétricos}
\label{sec:certificados_conjuntos-dosimetricos}
\inserirPDF{./anexos/PDFs/Certificados de Calibração/Conjunto Dosimetrico/CJ Dosimetrico 1.pdf}
%\inserirPDF{./anexos/PDFs/Certificados de Calibração/Conjunto Dosimetrico/CJ Dosimetrico 2.pdf}
%\inserirPDF{./anexos/PDFs/Certificados de Calibração/Conjunto Dosimetrico/Camara Poco.pdf}

\pagebreak
\section{Certificados de Calibração dos Demais Ítens}
\label{sec:certificados_demais}
\inserirPDF{./anexos/PDFs/Certificados de Calibração/Outros/Barometro.pdf}
%\inserirPDF{./anexos/PDFs/Certificados de Calibração/Outros/Nivel.pdf}
%\inserirPDF{./anexos/PDFs/Certificados de Calibração/Outros/Regua.pdf}
%\inserirPDF{./anexos/PDFs/Certificados de Calibração/Outros/Termometro.pdf}