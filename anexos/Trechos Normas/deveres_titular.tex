\noindent Art. 14 O titular do Serviço de Radioterapia é o principal responsável pela aplicação das Resoluções da CNEN relativas à segurança e proteção radiológica na instalação.\\
\noindent Art. 15 O titular do Serviço de Radioterapia é responsável pela segurança e proteção radiológica de pacientes, equipe médica, indivíduos ocupacionalmente expostos e indivíduos do público e deve obrigatoriamente:
\begin{enumerate}[label=\Roman*.]
    \item adotar as providências necessárias relativas ao licenciamento do Serviço de Radioterapia, de acordo com as Resoluções da CNEN;
    \item assegurar que todas as fontes de radiação estejam adequadamente instaladas e protegidas e providenciar o imediato recolhimento das fontes de radiação fora de uso, conforme Norma específica da CNEN;
    \item comunicar imediatamente à CNEN:
    \begin{enumerate}
        \item a retirada de uso de qualquer fonte de radiação e sua subsequente 	guarda; e
        \item as situações de emergência que exijam da mesma a adoção de 	qualquer ação de proteção radiológica.
    \end{enumerate}
        \item comunicar imediatamente à CNEN e demais autoridades competentes sobre a ocorrência de dano, perda ou roubo de fonte de radiação;
        \item designar os seguintes profissionais para compor o corpo técnico do Serviço de Radioterapia:
    \begin{enumerate}
        \item um responsável técnico;
        \item um substituto do responsável técnico;
        \item um supervisor de proteção radiológica de radioterapia;
        \item um substituto do supervisor de proteção radiológica de radioterapia;
        \item um especialista em física médica de radioterapia; e
        \item a quantidade necessária e suficiente de técnicos, seja de nível 	superior ou de nível médio, qualificados para o exercício de suas funções 	específicas.
    \end{enumerate}
    \item garantir que haja um médico radioterapeuta e um especialista em física médica de radioterapia para cada 600 novos pacientes por ano no Serviço de Radioterapia;
    \item garantir que haja, no mínimo, dois técnicos em radioterapia, por turno, por máquina;
    \item disponibilizar os recursos necessários para:
    \begin{enumerate}
        \item garantir a calibração dos instrumentos de medição em laboratório 	de metrologia;
        \item realizar treinamento anual de indivíduos ocupacionalmente 	expostos tanto para atuação em situações normais de trabalho, quanto em 	situações de incidente ou acidente;
        \item  minimizar a probabilidade de ocorrência de acidentes;
        \item executar um programa de manutenção preventiva para as fontes 	de radiação, com a definição de procedimentos e periodicidade das ações a serem realizadas; e
        \item atuar em situações normais de trabalho bem como em situações 	de incidente ou acidente.
    \end{enumerate}
    \item estabelecer um Serviço de Proteção Radiológica de acordo com as Resoluções da CNEN;
    \item estabelecer um Serviço de Física Médica de acordo com recomendações nacionais ou internacionais;
    \item garantir que no Serviço de Radioterapia:
    \begin{enumerate}
        \item seja cumprido o plano de proteção radiológica aprovado pela 	CNEN;
        \item somente pessoal treinado e autorizado opere e manipule as fontes 	de radiação;
        \item existam instrumentos de medição e dispositivos de controle da 	qualidade das fontes de radiação utilizadas;
        \item exista um sistema computadorizado de planejamento de 	tratamento, regularizado junto à Agência Nacional de Vigilância Sanitária 	(ANVISA), para as práticas executadas;
        \item  exista um segundo sistema de cálculo de dose para verificação 	do planejamento de tratamento;
        \item  exista um sistema computadorizado de gerenciamento de 	informação dos pacientes com cadastro e apresentação da fotografia do 	paciente em todos os documentos relacionados ao tratamento, assim como no 	painel de controle das fontes de radiação durante o tratamento;
        \item seja estabelecido um programa de garantia da qualidade em 	radioterapia, segundo o disposto nesta Norma e especificados em outras 	normas nacionais e recomendações internacionais;
        \item exista a participação em programas de auditoria externa e 	independente de garantia da qualidade das fontes de radiação e de sistemas de 	planejamento conforme descrito na seção IV do capítulo III desta Norma;
        \item seja realizada a manutenção de equipamentos de teleterapia e 	braquiterapia de alta taxa de dose somente por profissional ou empresa 	legalmente habilitados para essa atividade, pelo Conselho Federal de 	Engenharia, Arquitetura e Agronomia (CONFEA) ou Conselho Regional de Engenharia, Arquitetura e Agronomia (CREA);
        \item seja realizada a remoção e a colocação de fonte selada em 	cabeçote de fonte de radiação de teleterapia por empresa legalmente habilitada, 	para essa atividade, pelo CONFEA ou CREA e na presença de inspetores da 	CNEN;
        \item seja realizada a remoção e a colocação de fontes seladas em 	equipamentos de braquiterapia de alta taxa de dose por empresa legalmente 	habilitada para essas atividades, pelo CONFEA ou CREA; e
        \item sejam mantidos assentamentos e apresentados relatórios à CNEN, 	relativos às atividades autorizadas, de acordo com os requisitos regulatórios das 	Resoluções da CNEN.
    \end{enumerate}
    \item garantir livre acesso aos inspetores da CNEN às instalações, 	equipamentos, materiais e registros, seus e/ou emitidos por seus contratados, 	bem como às atividades em curso que estejam incluídas no processo de 	licenciamento;
    \item submeter, quando solicitado pela CNEN, relatórios e informações que possibilitem determinar se uma autorização deve ser mantida, alterada, suspensa ou revogada;
    \item submeter à CNEN um novo Plano de Proteção Radiológica, ou complementação daquele já aprovado, antes da introdução de quaisquer modificações em dados cadastrais, em projetos ou procedimentos que possam alterar as condições de proteção radiológica do Serviço de Radioterapia ou que modifiquem sua cadeia de responsabilidades; e
    \item provisionar recursos financeiros para garantir o descomissionamento da instalação, quando de sua retirada de operação.
\end{enumerate}