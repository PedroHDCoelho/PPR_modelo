\noindent Art. 18 O supervisor de proteção radiológica na área específica de Radioterapia de um Serviço de Radioterapia e seu substituto devem ser profissionais igualmente certificados de acordo com a Norma CNEN NN 7.01 Certificação da Qualificação de Supervisores de Proteção Radiológica para atuar em radioterapia.
\noindent Art. 19 O supervisor de proteção radiológica somente pode assumir a responsabilidade por um único Serviço de Radioterapia.
\noindent Art. 20 O supervisor de proteção radiológica em exercício é o responsável pela aplicação prática das diretrizes e normas relativas à segurança e proteção radiológica do Serviço de Radioterapia e deve obrigatoriamente:
\begin{enumerate}[label=\Roman*.]
    \item assessorar o titular e o responsável técnico do Serviço de Radioterapia sobre todos os assuntos relativos à segurança e à proteção radiológica;
    \item elaborar, aplicar e revisar o plano de proteção radiológica com a frequência nele estabelecida;	
	\item fazer cumprir o plano de proteção radiológica aprovado pela CNEN nos itens relativos à proteção radiológica;
    \item elaborar, aplicar e supervisionar o programa de monitoração individual e de monitoração de área, bem como gerenciar a documentação dos registros gerados;
    \item disponibilizar mensalmente a cada indivíduo ocupacionalmente exposto os valores das doses resultantes de sua monitoração individual;
    \item elaborar e supervisionar os programas de treinamento anual em proteção radiológica dos indivíduos ocupacionalmente expostos do Serviço de Radioterapia bem como informar todos os profissionais da instalação sobre os riscos inerentes ao uso da radiação ionizante;
    \item supervisionar os trabalhos de manutenção e o funcionamento das fontes de radiação;
    \item acompanhar e supervisionar os procedimentos de retirada e colocação de fontes de radiação dos cabeçotes dos equipamentos de teleterapia e de equipamentos de braquiterapia de alta taxa de dose;
    \item manter calibrados os instrumentos de medição de proteção radiológica por laboratório de metrologia acreditado pela Rede Brasileira de Calibração;
    \item acompanhar as inspeções realizadas por inspetores da CNEN;
    \item notificar o titular do Serviço de Radioterapia sobre os requisitos de segurança e proteção radiológica que não estejam de acordo com o plano de proteção radiológica;
    \item notificar o titular sobre todos os quesitos que não estejam de acordo com Resoluções da CNEN; e
    \item comunicar a CNEN, no prazo máximo de trinta dias, quando do seu desligamento do Serviço de Radioterapia.
\end{enumerate}
\noindent Art. 21 O supervisor de proteção radiológica deve analisar os resultados de controles e monitorações individuais e de área, de medidas de segurança e proteção radiológica, calibração de instrumentos de medição de proteção radiológica e providenciar as devidas correções e/ou reparos.