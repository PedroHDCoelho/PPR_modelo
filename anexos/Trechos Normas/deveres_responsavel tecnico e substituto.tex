\noindent Art. 16 O responsável técnico por um Serviço de Radioterapia e seu substituto eventual devem obrigatoriamente:
\begin{enumerate}[label=\Roman*.]
    \item ter registro na CNEN conforme a Norma CNEN NN 6.01 Requisitos para o Registro de Pessoas Físicas para o Preparo, Uso e Manuseio de Fontes Radioativas.
    \par Parágrafo único. O responsável técnico só poderá responder por um único serviço de radioterapia.
\end{enumerate}
\noindent Art. 17 O responsável técnico do Serviço de Radioterapia e seu substituto eventual devem obrigatoriamente:
\begin{enumerate}[label=\Roman*.]
    \item garantir que nenhum paciente seja submetido a uma exposição médica a menos que esta seja prescrita por um médico radioterapeuta com qualificação certificada por sociedade reconhecida representativa da classe;
    \item garantir que todos os médicos do Serviço de Radioterapia tenham como premissa a obrigação de assegurar proteção e segurança na prescrição e na execução da exposição médica;	
    \item garantir que seja disponibilizado pessoal médico e de enfermagem em número suficiente, com formação e treinamento específicos para conduzir os procedimentos de radioterapia;
    \item garantir que todos os planejamentos de tratamento sejam realizados por um especialista em física médica de radioterapia ou sob a sua supervisão, impressos em papel, e com uma segunda assinatura por conferência;
    \item garantir que esteja presente um médico radioterapeuta na sala de tratamento durante os preparativos e entrega da dose terapêutica, no primeiro dia de tratamento;
    \item notificar o titular sobre todos os quesitos que não estejam de acordo com as Normas e Resoluções da CNEN;	
    \item comunicar à CNEN, no prazo máximo de trinta dias, quando do seu desligamento do Serviço de Radioterapia; e
    \item garantir consulta de acompanhamento semanal do paciente, por um médico radioterapeuta, durante todo o tratamento.
\end{enumerate}