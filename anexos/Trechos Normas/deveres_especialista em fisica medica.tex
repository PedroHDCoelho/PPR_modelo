\noindent Art. 22 O especialista em física médica de radioterapia de um Serviço de Radioterapia deve obrigatoriamente possuir:
\begin{enumerate}[label=\Roman*.]
    \item I - titulação de especialista em física médica de radioterapia outorgado por instituição ou associação de referência nacional na área de radioterapia; e
    \item II - registro na CNEN, conforme a Norma CNEN NN 6.01 Requisitos para o Registro de Pessoas Físicas para o Preparo, Uso e Manuseio de Fontes Radioativas, ou outra que vier a substituí-la.
\end{enumerate}
\noindent Art. 23 Em Serviços de Radioterapia que tratam menos de 600 novos pacientes por ano, o especialista em física médica de radioterapia pode acumular a função de supervisor de proteção radiológica, desde que seja certificado pela CNEN para essa função. Parágrafo único. Todo serviço de radioterapia deve contar com a presença integral de, no mínimo, um físico médico.	
\noindent Art. 24 O especialista em física médica de radioterapia deve, obrigatoriamente:
\begin{enumerate}[label=\Roman*.]
\item conduzir:
    \begin{enumerate}
        \item testes pré-operacionais e de comissionamento das fontes de 	radiação e de sistemas de planejamento de tratamento;
        \item dosimetria periódica das fontes de radiação, segundo protocolos 	de dosimetria nacionais ou internacionais vigentes, descrito no plano de 	proteção radiológica;
        \item programa de controle da qualidade dos instrumentos de medição, 	fontes de radiação, sistemas de planejamento e acessórios de radioterapia;
        \item planejamento de tratamentos terapêuticos, conforme orientação 	do responsável técnico e equipe médica do Serviço de Radioterapia;	
        \item controle da qualidade dos tratamentos terapêuticos; e
        \item programas de treinamento em física médica dos indivíduos ocupacionalmente expostos, com periodicidade máxima de dois anos;
    \end{enumerate}
\item manter os sistemas de medição calibrados por laboratório de metrologia acreditado pela Rede Brasileira de Calibração, conforme descrito na seção IV do capítulo IV desta Norma;
\item auxiliar o responsável técnico na implementação de novas técnicas de tratamento em radioterapia;	
\item notificar o titular, o responsável técnico e o supervisor de proteção radiológica sobre todos os itens que não estejam de acordo com as normas e Resoluções da CNEN;
\item comunicar a CNEN, no prazo máximo de trinta dias, quando do seu desligamento do Serviço de Radioterapia; e
\item estar presente na sala de tratamento durante os preparativos e entrega da dose terapêutica, no primeiro dia de tratamento.
\end{enumerate}