%%
% ESSE TEMPLATE FOI DESENVOLVIDO POR PEDRO HENRIQUE DIAS COELHO COM OBJETIVO DE MELHORAR A ARQUITETURA E CONSTRUÇÃO DE RELATÓRIO DE PROTEÇÃO RADIOLÓGICA. SEU USO É DESTINADO A SUPERVISORES DE PROTEÇÃO RADIOLÓGICA E SEUS SUBSTITUTOS. O COMPARTILHAMENTO DESSA ARQUITETURA É DE USO LIVRE, MAS O AUTOR NÃO SE RESPONSABILIZA PELO CONTEÚDO INSERIDO E NEM MESMO DAS CONSEQUENCIAS LEGAIS COMO CONSEQUENCIA DO USO INCORRETO DESSE MODELO. 

% PEDROHDCOELHO@GMAIL.COM
% (11) 9 9347 4500

%%

\documentclass[
    article,
	12pt,				% tamanho da fonte
	%openright,			% capítulos começam em pág ímpar (insere página vazia caso preciso)
	oneside,			% para impressão em verso e anverso. Oposto a oneside
    chapter=TITLE,		% títulos de capítulos convertidos em letras maiúsculas
	section=Title,		% títulos de seções convertidos em letras maiúsculas
	subsection=Title,	% títulos de subseções convertidos em letras maiúsculas
	%subsubsection=TITLE,% títulos de subsubseções convertidos em letras maiúsculas
	brazil				% o último idioma é o principal do documento
	]{abntex2}
\usepackage[
    a4paper,
    total={170mm,225mm},
    includehead,
    headheight=30mm,
    voffset=-1in
]{geometry}

% ---
% Pacotes básicos 
% ---
\usepackage{lipsum}				% Para geração de dummy text
\usepackage{fancyhdr}           % Para cabecalho
\usepackage{helvet}             % Fonte de Texto
\usepackage[T1]{fontenc}		% Selecao de codigos de fonte.
\usepackage[utf8]{inputenc}		% Codificacao do documento (conversão automática dos acentos)
\usepackage{lastpage}			% Usado pela Ficha catalográfica
\usepackage{indentfirst}		% Indenta o primeiro parágrafo de cada seção.
\usepackage{nicematrix}         % Tabelas personalizadas
\usepackage{xcolor}             % Controle das cores aprimoradas
\usepackage{color}				% Controle das cores
\usepackage{graphicx}			% Inclusão de gráficos
\usepackage{subcaption}         % Colocar imagens subfigure
\usepackage{placeins}           % Usar o FloatBarrier nas equacoes
\usepackage{microtype} 			% Para melhorias de justificação
\usepackage{pdfpages}           % Colocar arquivos PDF no documento

% ---
% Oacotes adicionados
% ---
\usepackage{enumitem}
\usepackage{enumerate}
\usepackage{csvsimple}
\usepackage[most]{tcolorbox}
\usepackage{amsmath}
\usepackage{amssymb,amsfonts,amsthm}
\usepackage{setspace}

% ---
% Pacotes de citações
% ---
\usepackage{biblatex}           % Imports biblatex
\addbibresource{PPR.bib}
\nocite{nn301}
\nocite{pr301_04}
\nocite{nn602}
\nocite{tecdoc1151}
\nocite{trs398}
\nocite{sevrra}
\nocite{joana2018risk}

% ---
% Configurações de aparência do PDF final
% ---


\numberwithin{figure}{chapter}
\numberwithin{table}{chapter}
\makeatletter
\hypersetup{
		colorlinks=true,       		% false: boxed links; true: colored links
    	linkcolor=blue,          	% color of internal links
    	citecolor=blue,        		% color of links to bibliography
    	filecolor=blue,      		% color of file links
		urlcolor=blue,
		bookmarksdepth=4
}

\setlength{\parindent}{1.3cm} % O tamanho do parágrafo
\setlength{\parskip}{0.2cm}  % Controle do espaçamento entre um parágrafo e outro - tente também \onelineskip

\makeatother
\makeindex

\begin{document}
\selectlanguage{brazil}
\frenchspacing % Retira espaço extra obsoleto entre as frases.

\pretextual
%%
% ESSE TEMPLATE FOI DESENVOLVIDO POR PEDRO HENRIQUE DIAS COELHO COM OBJETIVO DE MELHORAR A ARQUITETURA E CONSTRUÇÃO DE RELATÓRIO DE PROTEÇÃO RADIOLÓGICA. SEU USO É DESTINADO A SUPERVISORES DE PROTEÇÃO RADIOLÓGICA E SEUS SUBSTITUTOS. O COMPARTILHAMENTO DESSA ARQUITETURA É DE USO LIVRE, MAS O AUTOR NÃO SE RESPONSABILIZA PELO CONTEÚDO INSERIDO E NEM MESMO DAS CONSEQUENCIAS LEGAIS COMO CONSEQUENCIA DO USO INCORRETO DESSE MODELO. 

% PEDROHDCOELHO@GMAIL.COM
% (11) 9 9347 4500

%%

% --- QUANTO A NATUREZA DO DOCUMENTO
\def\titulo{Plano de Proteção Radiológica\\****}
\def\autor{Departamento de Física Médica}
\def\local{Goiânia}
\def\ano{ANO}
\def\mes{MES}
\def\instituicao{**** - ****}
\def\tipotrabalho{Plano de Proteção Radiológica}
% O preambulo deve conter o tipo do trabalho, o objetivo, 
% o nome da instituição e a área de concentração 
\def\preambulo{O Plano de Proteção Radiológica compilado documento que se segue, tem por objetivo descrever deveres e diretos de todos os Indivíduos Ocupacionalmente Expostos e também diretrizes, projetos e bem-feiturias realizados por parte do \instituicao, para que as práticas sejam realizadas em consonância com as normas da CNEN.}
% ---------------------------------------------------------------------------------------------------

% --- QUANTO A PROTECAO RADIOLOGICA
\def\titularA{TITULAR VAI AQUI}
\def\titularACPF{****}

\def\titularB{TITULAR VAI AQUI}
\def\titularBCPF{****}

\def\supervisor{SUPERVISOR VAI AQUI}
\def\supervisorRT{****}
\def\supervisorRA{****}

\def\subsSupervisor{SUBSTITUTO SUPERVISOR VAI AQUI}
\def\subsSupervisorRT{****}
\def\subsSupervisorRA{****}

\def\rt{RT VAI AQUI}
\def\rtCRM{****}
\def\rtCB{****}

\def\subsRt{SUBSTITUTO RT VAI AQUI}
\def\subsRtCRM{****}
\def\subsRtCB{****}
% ---------------------------------------------------------------------------------------------------

% --- QUANTO A INSTALACAO
\def\instalacaoNOME{NOME DA INSTALAÇÃO VAI AQUI}
\def\instalacaoMATRICULA{****}
\def\instalacaoCNPJ{****}
\def\instalacaoRUA{****}
\def\instalacaoCOMPLEMENTO{****}
\def\instalacaoBAIRRO{****}
\def\instalacaoCIDADE{****}
\def\instalacaoUF{{****}}
\def\instalacaoCEP{****}
\def\instalacaoTELEFONE{****}
\def\instalacaoGRUPO{****}
\def\instalacaoSUBGRUPO{****}
% ---------------------------------------------------------------------------------------------------

% --- QUANTO A FORMATAÇÃO E PADRÃO DE CORES
% --- mude seu padrao de cores aqui
\definecolor{color1}{RGB}{46,108,150}
\definecolor{color2}{RGB}{91, 177, 169}

% --- 
% Formato de tabelas e negrito colorido
\newcolumntype{x}[1]{>{\centering\arraybackslash\vspace{0pt}}p{#1}}
\newcolumntype{y}[1]{>{\arraybackslash\vspace{0pt}}p{#1}}
\newcommand{\COLORtextbf}[1]{\textbf{\color{color1}#1}}

\def\colorPDF{color2}
\def\colorTABLE{color2}

% --- 
\renewcommand{\thesection}{\arabic{chapter}.{\arabic{section}}}
\renewcommand{\thesection}{\arabic{chapter}.{\arabic{section}}}
\renewcommand{\familydefault}{\sfdefault}

% --- 
% Inserir PDFs
\def\escalaPDF{0.9}
\newcommand{\inserirPDF}[1]{
    \begin{tcbraster}[raster columns=1,colframe = \colorPDF,colback = white,colbacktitle = \colorPDF!100,fonttitle = \small\ttfamily,boxsep = 0pt,toptitle = 1mm,bottomtitle = 1mm,drop lifted shadow,center title,title = {\imagename\ [\imagepage]}]
        \centering \tcbincludepdf[scale=\escalaPDF , graphics orientation=portrait]{#1}
    \end{tcbraster}
}
% ---------------------------------------------------------------------------------------------------
%%
% ESSE TEMPLATE FOI DESENVOLVIDO POR PEDRO HENRIQUE DIAS COELHO COM OBJETIVO DE MELHORAR A ARQUITETURA E CONSTRUÇÃO DE RELATÓRIO DE PROTEÇÃO RADIOLÓGICA. SEU USO É DESTINADO A SUPERVISORES DE PROTEÇÃO RADIOLÓGICA E SEUS SUBSTITUTOS. O COMPARTILHAMENTO DESSA ARQUITETURA É DE USO LIVRE, MAS O AUTOR NÃO SE RESPONSABILIZA PELO CONTEÚDO INSERIDO E NEM MESMO DAS CONSEQUENCIAS LEGAIS COMO CONSEQUENCIA DO USO INCORRETO DESSE MODELO. 

% PEDROHDCOELHO@GMAIL.COM
% (11) 9 9347 4500

%%

\begin{center}
    
    {\large\bfseries \instituicao} \\
    \vfill
    
    \large\bfseries{\titulo} \\
    \vfill
    
    \large\bfseries{\local \\ \ano}
\end{center}

\normalsize



%%
% ESSE TEMPLATE FOI DESENVOLVIDO POR PEDRO HENRIQUE DIAS COELHO COM OBJETIVO DE MELHORAR A ARQUITETURA E CONSTRUÇÃO DE RELATÓRIO DE PROTEÇÃO RADIOLÓGICA. SEU USO É DESTINADO A SUPERVISORES DE PROTEÇÃO RADIOLÓGICA E SEUS SUBSTITUTOS. O COMPARTILHAMENTO DESSA ARQUITETURA É DE USO LIVRE, MAS O AUTOR NÃO SE RESPONSABILIZA PELO CONTEÚDO INSERIDO E NEM MESMO DAS CONSEQUENCIAS LEGAIS COMO CONSEQUENCIA DO USO INCORRETO DESSE MODELO. 

% PEDROHDCOELHO@GMAIL.COM
% (11) 9 9347 4500

%%

\begin{center}

    {\large \instituicao}\\
    \vspace{8cm}
    {\Large \textsc\textbf{{\tipotrabalho} }\\}
    \vspace{1cm}
    \hspace{.45\linewidth}
    \begin{minipage}{.50\linewidth}
            \textbf{\preambulo}
    \end{minipage}

    \vspace{2cm}
    \vfill
    {\large \local, \mes \hspace{1pt} de \ano}
\end{center}
%%
% ESSE TEMPLATE FOI DESENVOLVIDO POR PEDRO HENRIQUE DIAS COELHO COM OBJETIVO DE MELHORAR A ARQUITETURA E CONSTRUÇÃO DE RELATÓRIO DE PROTEÇÃO RADIOLÓGICA. SEU USO É DESTINADO A SUPERVISORES DE PROTEÇÃO RADIOLÓGICA E SEUS SUBSTITUTOS. O COMPARTILHAMENTO DESSA ARQUITETURA É DE USO LIVRE, MAS O AUTOR NÃO SE RESPONSABILIZA PELO CONTEÚDO INSERIDO E NEM MESMO DAS CONSEQUENCIAS LEGAIS COMO CONSEQUENCIA DO USO INCORRETO DESSE MODELO. 

% PEDROHDCOELHO@GMAIL.COM
% (11) 9 9347 4500

%%

\begin{folhadeaprovacao}

    \begin{center}{\large \instituicao}\end{center}
    
    Este documento foi julgada adequada como Plano de Proteção Radiológica, sendo aprovada em sua forma final pelos seguintes responsáveis legais pelo cumprimento das medidas de proteção radiológica:
    
        \vfill
        \assinatura{\titularA\\ CPF: \titularACPF}
        \assinatura{\titularB\\ CPF: \titularBCPF}
        \vfill

        \assinatura{\supervisor \\Supervisor de Radioproteção\\CNEN RT \supervisorRT}
        \assinatura{\subsSupervisor \\ Substituto do Supervisor de Radioproteção\\CNEN RT \subsSupervisorRT}
        \vspace{1.5cm}
    
        \begin{center}\large \local, \mes \hspace{1pt} de \ano\end{center}
  
\end{folhadeaprovacao}

\pdfbookmark[0]{\listfigurename}{lof}
\listoffigures*
\clearpage

\pdfbookmark[0]{\listtablename}{lot}
\listoftables*
\clearpage

\pdfbookmark[0]{\contentsname}{toc}
\tableofcontents*
\clearpage

\fancypagestyle{fancy}{%
    \fancyhf{}
    \renewcommand{\headrulewidth}{1pt}
    \renewcommand{\footrulewidth}{0pt}
    \fancyhead[R]{\includegraphics[width=\headheight]{anexos/imagens/logo-radiofis.png}}
    %\fancyfoot[R]{\thepage}
}

\textual
\pagestyle{fancy}
\fancyhead{}
\fancyhead[RO,RE]{\includegraphics[width=4cm]{anexos/imagens/logo-radiofis.png}}

\pagebreak
\chapter{Identificação da Instalação}
O \instalacaoNOME (Número de matrícula CNEN \instalacaoMATRICULA) inscrita no C.N.P.J. sob o nº \instalacaoCNPJ, encontra-se situado à \instalacaoRUA, \instalacaoCOMPLEMENTO – \instalacaoBAIRRO – \instalacaoCIDADE – \instalacaoUF, CEP \instalacaoCEP, Telefone para contato: \instalacaoTELEFONE.

O Horário de tratamento de pacientes é das 07:00h às 19:00h, de segunda a sexta.\vspace{2cm}
\chapter{Objetivo da Instalação e Descrição das Práticas}
A citada instituição tem por objetivo o Tratamento Oncológico com Radioterapia.
\vspace{2cm}
\chapter{Estrutura Organizacional}
Vide anexo \ref{ch:estrutura_organizacional}.\vspace{2cm}
\chapter{Descrição dos IOEs}
Vide anexo \ref{ch:descricao_ioe} e suas responsabilidades descrito no anexo \ref{ch:descricao_responsabilidade}.\vspace{2cm}
\chapter{Classificação da Instalação}
De acordo com a Norma CNEN-NN-6.02, resolução 261/20, de Maio de 2020, a instalação é classificada com do GRUPO \instalacaoGRUPO, SUBGRUPO \instalacaoSUBGRUPO.\vspace{2cm}
\chapter{Classificação e Descrição das Áreas}
As áreas da instalação são classificadas como Áreas Controladas ou Áreas Livres,
de acordo com o diagrama apresentado no anexo \ref{ch:classificao_descricao_area}.\vspace{2cm}
\chapter{Mecanismos e Sistemas de Controle de Acesso de Acesso às Áreas da Instalação}
A Instalação dispõe de Barreiras Físicas destinadas a controlar o acesso de pessoas às áreas classificadas como Áreas Controladas. Além da supervisão de profissionais nas proximidades dos acessos, os mesmos contam com Identificação Visível do Símbolo Internacional de Radiação Ionizante e são mantidas fechadas, porém não trancadas, a fim de não dificultar a circulação dos IOE. Os indivíduos do público que acompanham os pacientes em tratamento são obrigatoriamente Identificados com crachás/etiquetas, para que seja orientada a permanência destes indivíduos apenas nas Áreas Livres ou sob supervisão em Áreas Controladas, ao auxiliar na condução dos pacientes à sala de tratamento.\vspace{2cm}
\chapter{Descrição dos Equipamentos e Fontes Emissoras de Radiação Ionizante}
Vide anexo \ref{ch:descricao_equipamentos_fontes-emissoras}.\vspace{2cm}
\chapter{Descrição das Fontes de Referência}
Vide anexo \ref{ch:descricao_fonte-referencia}.\vspace{2cm}
\chapter{Descrição dos Instrumentos de Dosimetria}
Vide anexo \ref{ch:descricao_instrumentos-dosimetria}.\vspace{2cm}
\chapter{Descrição dos Monitores de Área}
Vide anexo \ref{ch:descricao_monitores}.\vspace{2cm}
\chapter{Descrição do Programa de Garantia da Qualidade}
A Instalação dispõe de planilhas e gabaritos próprios para a realização de testes e
controles de qualidade, mecânicos e dosimétricos, do Acelerador Linear.

\section{Planilha de Controle e resultados}
Vide anexo \ref{ch:programa_garantia_qualidade} e seção \ref{sec:planilhas}.

\section{Testes Realizados, Periodicidade e Tolerâncias}
Vide anexo \ref{ch:programa_garantia_qualidade} e seção \ref{sec:teste_tolerancia}.

\section{Planejamento de Análise de Riscos}
A Instalação analisa os riscos inerentes à prática da atividade de Radioterapia através da ferramenta SEVRRA, constando os últimos resultados no anexo \ref{ch:programa_garantia_qualidade} e seção \ref{sec:sevrra}.\vspace{2cm}
\chapter{Sistemas de Planejamento}
Vide anexo \ref{ch:sistema_planejamento}.\vspace{2cm}
\chapter{Técnicas de Tratamento}
Vide anexo \ref{ch:tecnicas_tratamento}.\vspace{2cm}
\chapter{Cálculo de Barreiras e Estimativa de Dose}
Vide anexo \ref{ch:calculo_barreiras-estimativa_dose}.\vspace{2cm}
\chapter{Monitoração Individual}
A Instalação possui contrato de prestação de serviço de Monitoração Individual, de acordo com os termos constantes no anexo \ref{ch:monitoracao_individual}.\vspace{2cm}
\chapter{Monitoração de Áreas}
A instalação não possui áreas classificadas como Área Supervisionada. O Monitoramento de Áreas se dá de acordo com a periodicidade dos Levantamentos Radiométricos. Estes visam garantir os níveis de radiação preconizados no Cálculo de Barreiras referentes às Áreas Livres ou Controladas.\vspace{2cm}
\chapter{Gerência de Rejeitos Radioativos}
Vide anexo \ref{ch:gerencia_rejeitos_radioativos}\vspace{2cm}
\chapter{Descrição do Controle Médico dos IOEs - ASO}
Vide anexo \ref{ch:descricao_controle-medico}.\vspace{2cm}
\chapter{Programa de Treinamento em Proteção Radiológica para os IOEs}
Vide anexo \ref{ch:programa_treinamento_pr}.\vspace{2cm}
\chapter{Programa de Educação Continuada para os IOEs}
Vide anexo \ref{ch:programa_educacao_continuada}.\vspace{2cm}
\chapter{Níveis Operacionais e Restrições}
A Instalação segue as preconizações da CNEN, definidas na Posição Regulatória 3.01/004 para estabelecer os Níveis Operacionais:

\textit{“O nível de registro para monitoração individual mensal de IOE é de 0,10 mSv para dose efetiva: todas as doses maiores ou iguais a 0,10 mSv devem ser registradas. Níveis operacionais para fins de registro de monitoração em períodos inferiores ou superiores ao período mensal devem ser submetidos à aprovação da CNEN. O nível de investigação para monitoração individual de IOE deve ser, para dose efetiva, 6 mSv por ano ou 1 mSv em qualquer mês. Para dose equivalente, o nível de investigação para pele, mãos e pés é de 150 mSv por ano ou 20 mSv em qualquer mês. Para o cristalino, o nível de investigação é 6 mSv por ano ou 1 mSv em qualquer mês. Para fins de investigação, níveis operacionais em períodos de monitoração inferiores ou superiores ao período mensal devem ser submetidos à CNEN.”}

No caso de IOEs que comprovem situação de gestação, os mesmos são afastados de funções que podem oferecer risco de exposição e são encaminhadas para desempenhar outras tarefas em Áreas Livres na Instalação.\vspace{2cm}
\chapter{Procedimentos de Emergência}
Vide anexo \ref{ch:procedimentos_emergencia}.
\pagebreak


\phantompart
\postextual
\printbibliography


\begin{anexosenv}
    % Imprime uma página indicando o início dos anexos
    \partanexos
    
    \chapter{Estrutura Organizacional}
\label{ch:estrutura_organizacional}

\noindent \textbf{Titular da Instalação}\\
\titularA - CPF \titularACPF\\
 
\noindent \textbf{Identificação do SPR, Responsável Técnico e seus Substitutos}\\
\textbf{SPR}: \supervisor\\
RT - \supervisorRT\\
RA - \supervisorRA \\
\\
\textbf{Responsável Técnico:} \rt \\
CRM–GO: \rtCRM\\
CB - \rtCB \\
\\
\textbf{Substº. do SPR:} \subsSupervisor\\
RT - \subsSupervisorRT \\
RA – \subsSupervisorRA \\
\\
\textbf{Substº. Do Responsável Técnico}: \subsRt\\
CRM–GO: \subsRtCRM\\
CB – \subsRtCB
\pagebreak
    \chapter{Descrição dos IOEs}
\label{ch:descricao_ioe}

\section{Médicos Radio-Oncologista}
Os profissionais Médicos Radio-Oncologistas têm como Responsabilidades mínimas no serviço:

\begin{itemize}
    \item Cuidado integral com os pacientes
    \item Prescrever todo o tratamento e qualquer alteração que venha a ser necessária durante o mesmo na ficha técnica e prontuário, incluindo: Dose Total, Fracionamento e Dose Diária.
    \item Escolher a isodose de tratamento ou indicativo de cobertura do Volume-Alvo
    \item Avaliar o planejamento radioterápico tridimensional e assinar a aprovação do caso
    \item Verificar as imagens de simulação e tratamento, fazendo possíveis correções.
\end{itemize}

São os Radio-oncocologistas da instalação:

\begin{table}[!ht]
    \centering
    \resizebox{\linewidth}{!}{
    \scriptsize
    \begin{NiceTabular}{|p{6cm}|p{1.5cm}|p{1.5cm}|p{3cm}|}\hline%
    \rowcolor{\colorTABLE}
        \color{white}\bfseries Nome & \color{white}\bfseries CRM & \color{white}\bfseries RB  & \color{white}\bfseries Carga Horária\\\hline\hline
        \csvreader[head to column names, late after line = \\\hline]{./anexos/CSVs/medicos.csv}{}{%
        \Nome & \CRM & \CB & \Carga }%
        \hline
    \end{NiceTabular}
    }
    \caption{Equipe de Radio-Oncologistas.}
    \label{tab:radiooncologistas}
\end{table}

\section{Físicos Médicos}
Os Físicos Médicos da Radioterapia têm como responsabilidades mínimas no serviço:
\begin{itemize}
    \item Verificação das calibrações do equipamento para assegurar total segurança no tratamento.
    \item Assegurar que a dose entregue no tratamento esteja de acordo com a Prescrição Radioterapêutica.
    \item Promover treinamento teórico e prático aos técnicos operadores do acelerador linear.
    \item Reportar ao médico quaisquer incidentes ocorridos que possam afetar o tratamento ou a segurança do paciente e demais colaboradores.
    \item Manter um programa de dosimetria para assegurar a qualidade dos tratamentos.
    \item Realizar cálculos de dose no Sistema de Planejamento Tridimensional de acordo com a prescrição radioterapêutica.
\end{itemize}

Em específico, remete ao SPR a responsabilidade de:
\begin{itemize}
    \item Assegurar que as normas de proteção radiológicas sejam cumpridas.
\end{itemize}

São os físicos médicos da instalação:

\begin{table}[!h]
    \centering
    \resizebox{\linewidth}{!}{
    \scriptsize
    \begin{NiceTabular}{|p{7cm}|p{1cm}|p{1cm}|p{3cm}|}\hline%
    \rowcolor{\colorTABLE}
        \color{white}\bfseries Nome & \color{white}\bfseries RT & \color{white}\bfseries RA  & \color{white}\bfseries Carga Horária\\\hline\hline
        \csvreader[head to column names, late after line = \\\hline]{./anexos/CSVs/fisicos_medicos.csv}{}{%
        \Nome & \RT & \RA & \Carga }%
        \hline
    \end{NiceTabular}
    }
    \caption{Equipe de Físicos Médicos.}
    \label{tab:fisicos_medicos}
\end{table}

\section{Técnicos em Radioterapia}
Competem aos Técnicos em Radioterapia as seguintes responsabilidades mínimas no serviço:
\begin{itemize}
    \item Realizar o tratamento do paciente, conforme determinado pelo Médico e pelo Físico Médico na ficha técnica.
    \item Chamar o físico ou médico responsável quando houver qualquer dúvida relacionada ao tratamento dos pacientes.
    \item Programar o aparelho para liberar as unidades monitoras corretas, bem como tamanhos de campo de tratamento e posicionamento de acessórios de imobilização de acordo com o planejamento radioterapêutico.
    \item Encaminhar as imagens de tratamento e simulação de todos os pacientes aos médicos para possíveis correções durante o curso do tratamento.
    \item Reportar ao físico, todo e qualquer incidente ocorrido durante o tratamento do paciente.
    \item Manter a sala do acelerador linear em condições de trabalho.
\end{itemize}

São os técnicos em radioterapia da instalação:

\begin{table}[!h]
    \centering
    \resizebox{\linewidth}{!}{
    \scriptsize
    \begin{NiceTabular}{|p{7cm}|p{2cm}|p{3cm}|}\hline%
    \rowcolor{\colorTABLE}
        \color{white}\bfseries Nome & \color{white}\bfseries CRTR & \color{white}\bfseries Carga Horária\\\hline\hline
        \csvreader[head to column names, late after line = \\\hline]{./anexos/CSVs/tecnicos.csv}{}{%
        \Nome & \CRTR  & \Carga }%
        \hline
    \end{NiceTabular}
    }
    \caption{Equipe de Técnicos em Radioterapia.}
    \label{tab:tecnicos}
\end{table}

\section{Dosimetrista}
Competem aos Dosimetristas as seguintes responsabilidades mínimas no serviço:

\begin{itemize}
    \item Realizar exames de Tomografia de Planejamento.
    \item Importação e Fusão de Imagens de Exames no Sistema de Planejamento Tridimensional.
    \item Delineamento de Órgãos em Risco e delimitação do Corpo no Sistema de Planejamento Tridimensional.
    \item Confecção de Máscaras termoplásticas e demais acessórios de imobilização e posicionamento, dentre eles: Vac-Loks, Rampa de Mama, Rampa de Tórax, Apoios de Joelhos e pés para acomodação do paciente durante exame de Tomografia de Planejamento.
\end{itemize}

É dosimetrista da instalação:

\begin{table}[!h]
    \centering
    \resizebox{\linewidth}{!}{
    \scriptsize
    \begin{NiceTabular}{|p{6cm}|p{3.5cm}|p{3cm}|}\hline%
    \rowcolor{\colorTABLE}
        \color{white}\bfseries Nome & \color{white}\bfseries Registro Profissional & \color{white}\bfseries Carga Horária\\\hline\hline
        \csvreader[head to column names, late after line = \\\hline]{./anexos/CSVs/dosimetristas.csv}{}{%
        \Nome & \Registro  & \Carga }%
        \hline
    \end{NiceTabular}
    }
    \caption{Equipe de Dosimetristas.}
    \label{tab:dosimetristas}
\end{table}
 
\section{Enfermagem}
Os Profissionais em Enfermagem e terão como responsabilidades mínimas na Instalação:
\begin{itemize}
    \item Prestação de assistência à saúde do paciente;
    \item Orientações aos pacientes sobre possíveis efeitos colaterais relacionados ao tratamento;
    \item Orientações aos pacientes sobre preparos e cuidados necessários para a realização de Exames de Tomografia para Planejamento e/ou Sessões de Tratamento;
    \item Prestar atendimento personalizado e humanizado, a fim de acolher e individualizar cada paciente;
    \item Trabalhar Indicadores de Eventos Adversos e Reações Adversas relacionadas aos tratamentos, para identificar caminhos de melhorias no fluxo de atendimento do paciente;
    \item Prestar socorros em casos de Emergências Médicas ou Acidentes, até a chegada de Profissional Médico e/ou Suporte Avançado para transporte do indivíduo;
    \item Auxiliar na realização de Exames Clínicos em consultórios médicos;
    \item Atender consultas periódicas de avaliação durante o curso de tratamento dos pacientes.
\end{itemize}

A equipe de enfermagem é composta por:

\begin{table}[!h]
    \centering
    \resizebox{\linewidth}{!}{
    \scriptsize
    \begin{NiceTabular}{|p{6cm}|p{3cm}|p{3cm}|}\hline%
    \rowcolor{\colorTABLE}
        \color{white}\bfseries Nome & \color{white}\bfseries COREN & \color{white}\bfseries Carga Horária\\\hline\hline
        \csvreader[head to column names, late after line = \\\hline]{./anexos/CSVs/enfermagem.csv}{}{%
        \Nome & \Registro  & \Carga }%
        \hline
    \end{NiceTabular}
    }
    \caption{Equipe de Enfermagem.}
    \label{tab:enfermagem}
\end{table}
\pagebreak
    \chapter{Descrição das Responsabilidades}
\label{ch:descricao_responsabilidade}
Nesta Instalação, seguimos a orientação disposta na norma CNEN NN 6.10, referentes às responsabilidades do Titular, bem como do Responsável Técnico, SPR (e seus respectivos substitutos) e Especialista(s) em Física da Radioterapia:

\section{Titular da Instalação}
\noindent Art. 14 O titular do Serviço de Radioterapia é o principal responsável pela aplicação das Resoluções da CNEN relativas à segurança e proteção radiológica na instalação.\\
\noindent Art. 15 O titular do Serviço de Radioterapia é responsável pela segurança e proteção radiológica de pacientes, equipe médica, indivíduos ocupacionalmente expostos e indivíduos do público e deve obrigatoriamente:
\begin{enumerate}[label=\Roman*.]
    \item adotar as providências necessárias relativas ao licenciamento do Serviço de Radioterapia, de acordo com as Resoluções da CNEN;
    \item assegurar que todas as fontes de radiação estejam adequadamente instaladas e protegidas e providenciar o imediato recolhimento das fontes de radiação fora de uso, conforme Norma específica da CNEN;
    \item comunicar imediatamente à CNEN:
    \begin{enumerate}
        \item a retirada de uso de qualquer fonte de radiação e sua subsequente 	guarda; e
        \item as situações de emergência que exijam da mesma a adoção de 	qualquer ação de proteção radiológica.
    \end{enumerate}
        \item comunicar imediatamente à CNEN e demais autoridades competentes sobre a ocorrência de dano, perda ou roubo de fonte de radiação;
        \item designar os seguintes profissionais para compor o corpo técnico do Serviço de Radioterapia:
    \begin{enumerate}
        \item um responsável técnico;
        \item um substituto do responsável técnico;
        \item um supervisor de proteção radiológica de radioterapia;
        \item um substituto do supervisor de proteção radiológica de radioterapia;
        \item um especialista em física médica de radioterapia; e
        \item a quantidade necessária e suficiente de técnicos, seja de nível 	superior ou de nível médio, qualificados para o exercício de suas funções 	específicas.
    \end{enumerate}
    \item garantir que haja um médico radioterapeuta e um especialista em física médica de radioterapia para cada 600 novos pacientes por ano no Serviço de Radioterapia;
    \item garantir que haja, no mínimo, dois técnicos em radioterapia, por turno, por máquina;
    \item disponibilizar os recursos necessários para:
    \begin{enumerate}
        \item garantir a calibração dos instrumentos de medição em laboratório 	de metrologia;
        \item realizar treinamento anual de indivíduos ocupacionalmente 	expostos tanto para atuação em situações normais de trabalho, quanto em 	situações de incidente ou acidente;
        \item  minimizar a probabilidade de ocorrência de acidentes;
        \item executar um programa de manutenção preventiva para as fontes 	de radiação, com a definição de procedimentos e periodicidade das ações a serem realizadas; e
        \item atuar em situações normais de trabalho bem como em situações 	de incidente ou acidente.
    \end{enumerate}
    \item estabelecer um Serviço de Proteção Radiológica de acordo com as Resoluções da CNEN;
    \item estabelecer um Serviço de Física Médica de acordo com recomendações nacionais ou internacionais;
    \item garantir que no Serviço de Radioterapia:
    \begin{enumerate}
        \item seja cumprido o plano de proteção radiológica aprovado pela 	CNEN;
        \item somente pessoal treinado e autorizado opere e manipule as fontes 	de radiação;
        \item existam instrumentos de medição e dispositivos de controle da 	qualidade das fontes de radiação utilizadas;
        \item exista um sistema computadorizado de planejamento de 	tratamento, regularizado junto à Agência Nacional de Vigilância Sanitária 	(ANVISA), para as práticas executadas;
        \item  exista um segundo sistema de cálculo de dose para verificação 	do planejamento de tratamento;
        \item  exista um sistema computadorizado de gerenciamento de 	informação dos pacientes com cadastro e apresentação da fotografia do 	paciente em todos os documentos relacionados ao tratamento, assim como no 	painel de controle das fontes de radiação durante o tratamento;
        \item seja estabelecido um programa de garantia da qualidade em 	radioterapia, segundo o disposto nesta Norma e especificados em outras 	normas nacionais e recomendações internacionais;
        \item exista a participação em programas de auditoria externa e 	independente de garantia da qualidade das fontes de radiação e de sistemas de 	planejamento conforme descrito na seção IV do capítulo III desta Norma;
        \item seja realizada a manutenção de equipamentos de teleterapia e 	braquiterapia de alta taxa de dose somente por profissional ou empresa 	legalmente habilitados para essa atividade, pelo Conselho Federal de 	Engenharia, Arquitetura e Agronomia (CONFEA) ou Conselho Regional de Engenharia, Arquitetura e Agronomia (CREA);
        \item seja realizada a remoção e a colocação de fonte selada em 	cabeçote de fonte de radiação de teleterapia por empresa legalmente habilitada, 	para essa atividade, pelo CONFEA ou CREA e na presença de inspetores da 	CNEN;
        \item seja realizada a remoção e a colocação de fontes seladas em 	equipamentos de braquiterapia de alta taxa de dose por empresa legalmente 	habilitada para essas atividades, pelo CONFEA ou CREA; e
        \item sejam mantidos assentamentos e apresentados relatórios à CNEN, 	relativos às atividades autorizadas, de acordo com os requisitos regulatórios das 	Resoluções da CNEN.
    \end{enumerate}
    \item garantir livre acesso aos inspetores da CNEN às instalações, 	equipamentos, materiais e registros, seus e/ou emitidos por seus contratados, 	bem como às atividades em curso que estejam incluídas no processo de 	licenciamento;
    \item submeter, quando solicitado pela CNEN, relatórios e informações que possibilitem determinar se uma autorização deve ser mantida, alterada, suspensa ou revogada;
    \item submeter à CNEN um novo Plano de Proteção Radiológica, ou complementação daquele já aprovado, antes da introdução de quaisquer modificações em dados cadastrais, em projetos ou procedimentos que possam alterar as condições de proteção radiológica do Serviço de Radioterapia ou que modifiquem sua cadeia de responsabilidades; e
    \item provisionar recursos financeiros para garantir o descomissionamento da instalação, quando de sua retirada de operação.
\end{enumerate}

\section{Responsável Técnico e Substituto}
\noindent Art. 16 O responsável técnico por um Serviço de Radioterapia e seu substituto eventual devem obrigatoriamente:
\begin{enumerate}[label=\Roman*.]
    \item ter registro na CNEN conforme a Norma CNEN NN 6.01 Requisitos para o Registro de Pessoas Físicas para o Preparo, Uso e Manuseio de Fontes Radioativas.
    \par Parágrafo único. O responsável técnico só poderá responder por um único serviço de radioterapia.
\end{enumerate}
\noindent Art. 17 O responsável técnico do Serviço de Radioterapia e seu substituto eventual devem obrigatoriamente:
\begin{enumerate}[label=\Roman*.]
    \item garantir que nenhum paciente seja submetido a uma exposição médica a menos que esta seja prescrita por um médico radioterapeuta com qualificação certificada por sociedade reconhecida representativa da classe;
    \item garantir que todos os médicos do Serviço de Radioterapia tenham como premissa a obrigação de assegurar proteção e segurança na prescrição e na execução da exposição médica;	
    \item garantir que seja disponibilizado pessoal médico e de enfermagem em número suficiente, com formação e treinamento específicos para conduzir os procedimentos de radioterapia;
    \item garantir que todos os planejamentos de tratamento sejam realizados por um especialista em física médica de radioterapia ou sob a sua supervisão, impressos em papel, e com uma segunda assinatura por conferência;
    \item garantir que esteja presente um médico radioterapeuta na sala de tratamento durante os preparativos e entrega da dose terapêutica, no primeiro dia de tratamento;
    \item notificar o titular sobre todos os quesitos que não estejam de acordo com as Normas e Resoluções da CNEN;	
    \item comunicar à CNEN, no prazo máximo de trinta dias, quando do seu desligamento do Serviço de Radioterapia; e
    \item garantir consulta de acompanhamento semanal do paciente, por um médico radioterapeuta, durante todo o tratamento.
\end{enumerate}
 
\section{SPR e Substituto}
\noindent Art. 18 O supervisor de proteção radiológica na área específica de Radioterapia de um Serviço de Radioterapia e seu substituto devem ser profissionais igualmente certificados de acordo com a Norma CNEN NN 7.01 Certificação da Qualificação de Supervisores de Proteção Radiológica para atuar em radioterapia.
\noindent Art. 19 O supervisor de proteção radiológica somente pode assumir a responsabilidade por um único Serviço de Radioterapia.
\noindent Art. 20 O supervisor de proteção radiológica em exercício é o responsável pela aplicação prática das diretrizes e normas relativas à segurança e proteção radiológica do Serviço de Radioterapia e deve obrigatoriamente:
\begin{enumerate}[label=\Roman*.]
    \item assessorar o titular e o responsável técnico do Serviço de Radioterapia sobre todos os assuntos relativos à segurança e à proteção radiológica;
    \item elaborar, aplicar e revisar o plano de proteção radiológica com a frequência nele estabelecida;	
	\item fazer cumprir o plano de proteção radiológica aprovado pela CNEN nos itens relativos à proteção radiológica;
    \item elaborar, aplicar e supervisionar o programa de monitoração individual e de monitoração de área, bem como gerenciar a documentação dos registros gerados;
    \item disponibilizar mensalmente a cada indivíduo ocupacionalmente exposto os valores das doses resultantes de sua monitoração individual;
    \item elaborar e supervisionar os programas de treinamento anual em proteção radiológica dos indivíduos ocupacionalmente expostos do Serviço de Radioterapia bem como informar todos os profissionais da instalação sobre os riscos inerentes ao uso da radiação ionizante;
    \item supervisionar os trabalhos de manutenção e o funcionamento das fontes de radiação;
    \item acompanhar e supervisionar os procedimentos de retirada e colocação de fontes de radiação dos cabeçotes dos equipamentos de teleterapia e de equipamentos de braquiterapia de alta taxa de dose;
    \item manter calibrados os instrumentos de medição de proteção radiológica por laboratório de metrologia acreditado pela Rede Brasileira de Calibração;
    \item acompanhar as inspeções realizadas por inspetores da CNEN;
    \item notificar o titular do Serviço de Radioterapia sobre os requisitos de segurança e proteção radiológica que não estejam de acordo com o plano de proteção radiológica;
    \item notificar o titular sobre todos os quesitos que não estejam de acordo com Resoluções da CNEN; e
    \item comunicar a CNEN, no prazo máximo de trinta dias, quando do seu desligamento do Serviço de Radioterapia.
\end{enumerate}
\noindent Art. 21 O supervisor de proteção radiológica deve analisar os resultados de controles e monitorações individuais e de área, de medidas de segurança e proteção radiológica, calibração de instrumentos de medição de proteção radiológica e providenciar as devidas correções e/ou reparos.
    
\section{Especialista em Física da Radioterapia}
\noindent Art. 22 O especialista em física médica de radioterapia de um Serviço de Radioterapia deve obrigatoriamente possuir:
\begin{enumerate}[label=\Roman*.]
    \item I - titulação de especialista em física médica de radioterapia outorgado por instituição ou associação de referência nacional na área de radioterapia; e
    \item II - registro na CNEN, conforme a Norma CNEN NN 6.01 Requisitos para o Registro de Pessoas Físicas para o Preparo, Uso e Manuseio de Fontes Radioativas, ou outra que vier a substituí-la.
\end{enumerate}
\noindent Art. 23 Em Serviços de Radioterapia que tratam menos de 600 novos pacientes por ano, o especialista em física médica de radioterapia pode acumular a função de supervisor de proteção radiológica, desde que seja certificado pela CNEN para essa função. Parágrafo único. Todo serviço de radioterapia deve contar com a presença integral de, no mínimo, um físico médico.	
\noindent Art. 24 O especialista em física médica de radioterapia deve, obrigatoriamente:
\begin{enumerate}[label=\Roman*.]
\item conduzir:
    \begin{enumerate}
        \item testes pré-operacionais e de comissionamento das fontes de 	radiação e de sistemas de planejamento de tratamento;
        \item dosimetria periódica das fontes de radiação, segundo protocolos 	de dosimetria nacionais ou internacionais vigentes, descrito no plano de 	proteção radiológica;
        \item programa de controle da qualidade dos instrumentos de medição, 	fontes de radiação, sistemas de planejamento e acessórios de radioterapia;
        \item planejamento de tratamentos terapêuticos, conforme orientação 	do responsável técnico e equipe médica do Serviço de Radioterapia;	
        \item controle da qualidade dos tratamentos terapêuticos; e
        \item programas de treinamento em física médica dos indivíduos ocupacionalmente expostos, com periodicidade máxima de dois anos;
    \end{enumerate}
\item manter os sistemas de medição calibrados por laboratório de metrologia acreditado pela Rede Brasileira de Calibração, conforme descrito na seção IV do capítulo IV desta Norma;
\item auxiliar o responsável técnico na implementação de novas técnicas de tratamento em radioterapia;	
\item notificar o titular, o responsável técnico e o supervisor de proteção radiológica sobre todos os itens que não estejam de acordo com as normas e Resoluções da CNEN;
\item comunicar a CNEN, no prazo máximo de trinta dias, quando do seu desligamento do Serviço de Radioterapia; e
\item estar presente na sala de tratamento durante os preparativos e entrega da dose terapêutica, no primeiro dia de tratamento.
\end{enumerate}
\pagebreak
    \chapter{Classificação e Descrição das Áreas da Instalação}
\label{ch:classificao_descricao_area}

Exemplo de Sinalização de acesso a Áreas Controladas como ilustra as figuras \ref{fig:trifolio}, \ref{fig:exemplo_acessos_subsolo} e \ref{fig:exemplo_acessos_1andar}.

\begin{figure}[!h]
    \centering
    \includegraphics[width=0.5\linewidth]{anexos//imagens//Classificação de Áreas/trifolio.png}
    \caption{Sinalização de Área Controla.}
    \label{fig:trifolio}
\end{figure}

\section{Proximidades do AL}
\label{sec:proximidades_6ex}
\inserirPDF{./anexos/PDFs/Classificação de Áreas/6EX.pdf}\pagebreak
    \chapter{Descrição dos Equipamentos e Fontes Emissoras de Radiação Ionizante}
\label{ch:descricao_equipamentos_fontes-emissoras}
A Instalação dispõe de (1) um Equipamento Gerador de Radiação Ionizante:

\begingroup
    \noindent \COLORtextbf{Acelerador Linear de Teleterapia}\\
    Fabricante: \COLORtextbf{Varian Medical Systems}\\
    Modelo: \COLORtextbf{Clinac 6EX}\\
    Nº de Série: \COLORtextbf{1254} \\
    Data de Fabricação: \COLORtextbf{****}\\
    Data de Aceite: \COLORtextbf{****}\\
    Energia nominal: \COLORtextbf{****}\\
    Tipo de Radiação: \COLORtextbf{****}\\
    Taxa de dose nominal: 100-600 cGy/min a 101,5cm da fonte, nas condições de referência (Campo 10x10 cm², 	profundidade de água 1,5cm) para Fótons de 6MV
\endgroup


\vfill\pagebreak
    \chapter{Descrição das Fontes de Referência}
\label{ch:descricao_fonte-referencia}
A Instalação dispõe de (1) uma Fonte de Referência Selada, como listado na tabela \ref{tab:fontes_referencia}, para aferição periódica do desempenho dos conjuntos dosimétricos:

\begin{table}[!ht]
    \centering
    \resizebox{\linewidth}{!}{
    \scriptsize
        \begin{NiceTabular}{|p{3cm}|p{3cm}|p{2cm}|p{2cm}|p{2cm}|p{2cm}|p{1cm}|}\hline %
        \rowcolor{\colorTABLE}
            \color{white}\bfseries Fonte Selada & \color{white}\bfseries Fabricante & \color{white}\bfseries Modelo  & \color{white}\bfseries Nº Série & \color{white}\bfseries Data Fabricação & \color{white}\bfseries Atividade Inicial & \color{white}\bfseries Tipo\\\hline\hline
            \csvreader[head to column names, late after line = \\\hline]{./anexos/CSVs/fontes_referencia.csv}{}%
            {\obj & \fabricante & \modelo & \nserie & \fabricacao & \atividade & \tipo}%
            \hline
        \end{NiceTabular}
    }
    \caption{Fonte de Referência.}
    \label{tab:fontes_referencia}
\end{table}

\vfill\pagebreak
    \chapter{Descrição dos Instrumentos de Dosimetria}
\label{ch:descricao_instrumentos-dosimetria}
A Instalação dispõe de (2) dois Conjuntos Dosimétricos, como listado na tabela \ref{tab:conjuntos_dosimetricos}, para realização de dosimetrias absolutas e os respectivos Certificados de Calibração mais recentes dos conjuntos dosimétricos e demais instrumentos de de medição, listados na tabela \ref{tab:instrumentos_medicao} estão dispostos nas seções \ref{sec:certificados_conjuntos-dosimetricos} e \ref{sec:certificados_demais}, nesse anexo.	:

\section{Câmaras de Ionização e Eletrômetros}
\label{sec:camaras_eletrometros}

\begin{table}[!h]
    \centering
    \resizebox{\linewidth}{!}{
    \scriptsize
    \begin{NiceTabular}{|p{5cm}|p{2.5cm}|p{2cm}|p{4cm}|}\hline%
    \rowcolor{\colorTABLE}
        \color{white}\bfseries Item & \color{white}\bfseries Fabricante & \color{white}\bfseries Modelo  & \color{white}\bfseries Nº Série\\\hline\hline
        \csvreader[head to column names, late after line = \\\hline]{./anexos/CSVs/conjunto_dosimetrico.csv}{}{%
        \obj & \fabricante & \modelo & \nserie }%
        \hline
    \end{NiceTabular}
    }
    \caption{Conjuntos dosimétricos.}
    \label{tab:conjuntos_dosimetricos}
\end{table}

  
\section{Termômetro(s), Barômetro(s), Nível(is), Régua(s) e Cronômetro(s)}
\label{sec:ter_bar_niv_reg_cro}
A Instalação dispõe de (1) um Termômetro, (1) um Barômetro, (1) um Nível Digital e (1) uma Régua Calibrados, além de (1) Cronômetro, listados na tabela \ref{tab:instrumentos_medicao}, e os respectivos Certificados de Calibração mais recentes dos Instrumentos a seguir encontram-se disponíveis no final desse anexo, na seção \ref{sec:certificados_demais}.

\begin{table}[!h]
    \centering
    \resizebox{\linewidth}{!}{
    \scriptsize
    \begin{NiceTabular}{|p{5cm}|p{2.5cm}|p{2cm}|p{2cm}|}\hline%
    \rowcolor{\colorTABLE}
        \color{white}\bfseries Item & \color{white}\bfseries Fabricante & \color{white}\bfseries Modelo  & \color{white}\bfseries Nº Série\\\hline\hline
        \csvreader[head to column names, late after line = \\\hline]{./anexos/CSVs/instrumentos_medicao.csv}{}{%
        \obj & \fabricante & \modelo & \nserie }%
        \hline
    \end{NiceTabular}
    }
    \caption{Outros instrumentos de medição.}
    \label{tab:instrumentos_medicao}
\end{table}

\pagebreak
\section{Fantomas}
\label{sec:fantomas}
A Instalação dispõe de (1) um fantoma de água para dosimetria absoluta, como listado na tabela \ref{tab:fantoma}:

\begin{table}[!h]
    \centering
    \resizebox{\linewidth}{!}{
    \scriptsize
    \begin{NiceTabular}{|p{5cm}|p{2.5cm}|p{2cm}|p{2.5cm}|p{2.5cm}|}\hline%
    \rowcolor{\colorTABLE}
        \color{white}\bfseries Item & \color{white}\bfseries Fabricante & \color{white}\bfseries Modelo  & \color{white}\bfseries Dimensões & \color{white}\bfseries Material\\\hline\hline
        \csvreader[head to column names, late after line = \\\hline]{./anexos/CSVs/fantoma.csv}{}{%
        \obj & \fabricante & \modelo & \dimensoes & \material }%
        \hline
    \end{NiceTabular}
    }
    \caption{Fantomas.}
    \label{tab:fantoma}
\end{table}
    
\section{Demais Equipamentos}
\label{sec:outros_eqp}
A Instalação dispõe de equipamentos específicos para realização de Controle de Qualidade, conforme listado na tabela \ref{tab:outros_detectores}:

\begin{table}[!h]
    \centering
    \resizebox{\linewidth}{!}{
    \scriptsize
    \begin{NiceTabular}{|p{3cm}|p{2.5cm}|p{2.5cm}|p{2cm}|p{2cm}|p{1cm}|}\hline%
    \rowcolor{\colorTABLE}
        \color{white}\bfseries Item & \color{white}\bfseries Fabricante & \color{white}\bfseries Modelo & \color{white}\bfseries Nº Série  & \color{white}\bfseries Data Fabricação & \color{white}\bfseries Tipo\\\hline\hline
        \csvreader[head to column names, late after line = \\\hline]{./anexos/CSVs/outros_detectores.csv}{}{%
        \obj & \fabricante & \modelo & \nserie & \data & \tipo }%
        \hline
    \end{NiceTabular}
    }
    \caption{Fantoma.}
    \label{tab:outros_detectores}
\end{table}

    
\section{Descrição dos Controles de Qualidade dos Equipamentos de Dosimetria}
Os equipamentos de dosimetria passam por procedimentos de aferições de desempenho trimestrais, contemplando testes de fuga do conjunto câmara-eletrômetro, medidas de reprodutibilidade e linearidade. Ademais, cada conjunto passa, conforme normas vigentes, por calibração em laboratório secundário bianualmente, porém nunca ambos ao mesmo tempo.
Os equipamentos descritos nas seções \ref{sec:camaras_eletrometros} e \ref{sec:ter_bar_niv_reg_cro}  passam por calibração em laboratório de metrologia anualmente, conforme normas vigentes.

\pagebreak
\section{Certificados de Calibração dos Conjuntos Dosimétricos}
\label{sec:certificados_conjuntos-dosimetricos}
\inserirPDF{./anexos/PDFs/Certificados de Calibração/Conjunto Dosimetrico/CJ Dosimetrico 1.pdf}
%\inserirPDF{./anexos/PDFs/Certificados de Calibração/Conjunto Dosimetrico/CJ Dosimetrico 2.pdf}
%\inserirPDF{./anexos/PDFs/Certificados de Calibração/Conjunto Dosimetrico/Camara Poco.pdf}

\pagebreak
\section{Certificados de Calibração dos Demais Ítens}
\label{sec:certificados_demais}
\inserirPDF{./anexos/PDFs/Certificados de Calibração/Outros/Barometro.pdf}
%\inserirPDF{./anexos/PDFs/Certificados de Calibração/Outros/Nivel.pdf}
%\inserirPDF{./anexos/PDFs/Certificados de Calibração/Outros/Regua.pdf}
%\inserirPDF{./anexos/PDFs/Certificados de Calibração/Outros/Termometro.pdf}\pagebreak
    \chapter{Descrição dos Monitores de Área}
\label{ch:descricao_monitores}

A Instalação dispõe de (2) dois monitores de área do tipo Geiger-Müller, listados na tabela \ref{tab:monitor_area}, e os respectivos Certificados de Calibração mais recentes encontram-se disponíveis nesse anexo na seção \ref{sec:certificados_monitores-area}.

\begin{table}[!h]
    \centering
    \resizebox{\linewidth}{!}{
    \scriptsize
    \begin{NiceTabular}{|p{5cm}|p{2.5cm}|p{2cm}|p{2cm}|}\hline%
    \rowcolor{\colorTABLE}
        \color{white}\bfseries Item & \color{white}\bfseries Fabricante & \color{white}\bfseries Modelo  & \color{white}\bfseries Nº Série\\\hline\hline
        \csvreader[head to column names, late after line = \\\hline]{./anexos/CSVs/monitor_de_area.csv}{}{%
        \obj & \fabricante & \modelo & \nserie }%
        \hline
    \end{NiceTabular}
    }
    \caption{Monitores de área.}
    \label{tab:monitor_area}
\end{table}
\vfill

\pagebreak
\section{Certificados de Calibração dos Monitores de Área}
\label{sec:certificados_monitores-area}

\inserirPDF{./anexos/PDFs/Certificados de Calibração/Monitor de Area/Monitor de Area 1.pdf}

%\inserirPDF{./anexos/PDFs/Certificados de Calibração/Monitor de Area/Monitor de Area 2.pdf}\pagebreak
    \chapter{Programa de Garantia da Qualidade}
\label{ch:programa_garantia_qualidade}

\section{Planilha de Controle de Qualidade e Resultados}
\label{sec:planilhas}


\clearpage
\section{Testes e Tolerâncias}
\label{sec:teste_tolerancia}
Os testes realizados no Acelerador Linear seguem as periodicidades e tolerâncias dispostas no TECDOC 1151. Os testes diários realizados estão listados nas tabelas \ref{tab:testes_diarios_seguranca}, \ref{tab:testes_diarios_mecanicos} e \ref{tab:testes_diarios_dosimetricos}.  Os testes mensais realizados estão listados nas tabelas \ref{tab:testes_mensais_seguranca}, \ref{tab:testes_mensais_mecanicos} e \ref{tab:testes_mensais_dosimetricos}. E  os testes anuais realizados estão listados nas tabelas \ref{tab:testes_anuais_seguranca}, \ref{tab:testes_anuais_mecanicos} e \ref{tab:testes_anuais_dosimetricos}.
\vfill

\pagebreak
\subsection{Diários}

\begin{table}[!h]
    \centering
    \begin{NiceTabular}{|p{8cm}|p{4cm}|}\hline%
    \rowcolor{color2!100}
        \color{white}\bfseries Teste & \color{white}\bfseries Tolerância \\\hline\hline
        \csvreader[head to column names, late after line = \\\hline]{./anexos/CSVs/testes/diarios/seguranca.csv}{}{%
        \teste & \tolerancia}%
        \hline
    \end{NiceTabular}
    \caption{Testes diários de segurança.}
    \label{tab:testes_diarios_seguranca}
\end{table}

\begin{table}[!h]
    \centering
    \begin{NiceTabular}{|p{8cm}|p{4cm}|}\hline%
    \rowcolor{color2!100}
        \color{white}\bfseries Teste & \color{white}\bfseries Tolerância \\\hline\hline
        \csvreader[head to column names, late after line = \\\hline]{./anexos/CSVs/testes/diarios/dosimetrico.csv}{}{%
        \teste & \tolerancia}%
        \hline
    \end{NiceTabular}
    \caption{Testes diários dosimétricos.}
    \label{tab:testes_diarios_dosimetricos}
\end{table}

\begin{table}[!h]
    \centering
    \begin{NiceTabular}{|p{8cm}|p{4cm}|}\hline%
    \rowcolor{color2!100}
        \color{white}\bfseries Teste & \color{white}\bfseries Tolerância \\\hline\hline
        \csvreader[head to column names, late after line = \\\hline]{./anexos/CSVs/testes/diarios/mecanico.csv}{}{%
        \teste & \tolerancia}%
        \hline
    \end{NiceTabular}
    \caption{Testes diários mecânicos.}
    \label{tab:testes_diarios_mecanicos}
\end{table}
\vfill

\pagebreak
\subsection{Mensais}

\begin{table}[h]
    \centering
    \begin{NiceTabular}{|p{8cm}|p{4cm}|}\hline%
    \rowcolor{color2!100}
        \color{white}\bfseries Teste & \color{white}\bfseries Tolerância \\\hline\hline
        \csvreader[head to column names, late after line = \\\hline]{./anexos/CSVs/testes/mensais/seguranca.csv}{}{%
        \teste & \tolerancia}%
        \hline
    \end{NiceTabular}
    \caption{Testes mensais de segurança.}
    \label{tab:testes_mensais_seguranca}
\end{table}

\begin{table}[h]
    \centering
    \begin{NiceTabular}{|p{8cm}|p{4cm}|}\hline%
    \rowcolor{color2!100}
        \color{white}\bfseries Teste & \color{white}\bfseries Tolerância \\\hline\hline
        \csvreader[head to column names, late after line = \\\hline]{./anexos/CSVs/testes/mensais/dosimetrico.csv}{}{%
        \teste & \tolerancia}%
        \hline
    \end{NiceTabular}
    \caption{Testes mensais dosimétricos.}
    \label{tab:testes_mensais_dosimetricos}
\end{table}

\begin{table}[h]
    \centering
    \begin{NiceTabular}{|p{8cm}|p{4cm}|}\hline%
    \rowcolor{color2!100}
        \color{white}\bfseries Teste & \color{white}\bfseries Tolerância \\\hline\hline
        \csvreader[head to column names, late after line = \\\hline]{./anexos/CSVs/testes/mensais/mecanico.csv}{}{%
        \teste & \tolerancia}%
        \hline
    \end{NiceTabular}
    \caption{Testes mensais mecânicos.}
    \label{tab:testes_mensais_mecanicos}
\end{table}
\vfill

\clearpage
\subsection{Anuais}

\begin{table}[h]
    \centering
    \begin{NiceTabular}{|p{8cm}|p{4cm}|}\hline%
    \rowcolor{color2!100}
        \color{white}\bfseries Teste & \color{white}\bfseries Tolerância \\\hline\hline
        \csvreader[head to column names, late after line = \\\hline]{./anexos/CSVs/testes/anuais/seguranca.csv}{}{%
        \teste & \tolerancia}%
        \hline
    \end{NiceTabular}
    \caption{Testes anuais de segurança.}
    \label{tab:testes_anuais_seguranca}
\end{table}

\begin{table}[h]
    \centering
    \begin{NiceTabular}{|p{8cm}|p{4cm}|}\hline%
    \rowcolor{color2!100}
        \color{white}\bfseries Teste & \color{white}\bfseries Tolerância \\\hline\hline
        \csvreader[head to column names, late after line = \\\hline]{./anexos/CSVs/testes/anuais/dosimetrico.csv}{}{%
        \teste & \tolerancia}%
        \hline
    \end{NiceTabular}
    \caption{Testes anuais dosimétricos.}
    \label{tab:testes_anuais_dosimetricos}
\end{table}

\begin{table}[h]
    \centering
    \begin{NiceTabular}{|p{8cm}|p{4cm}|}\hline%
    \rowcolor{color2!100}
        \color{white}\bfseries Teste & \color{white}\bfseries Tolerância \\\hline\hline
        \csvreader[head to column names, late after line = \\\hline]{./anexos/CSVs/testes/anuais/mecanico.csv}{}{%
        \teste & \tolerancia}%
        \hline
    \end{NiceTabular}
    \caption{Testes anuais mecânicos.}
    \label{tab:testes_anuais_mecanicos}
\end{table}

\clearpage
\subsection{Conjuntos Dosimétricos}
\label{sec:conjuntos_dosimtricos}

\clearpage
\section{SEVRRA}
\label{sec:sevrra}
\inserirPDF{./anexos/PDFs/SEVRRA/SEVRRA.pdf}
\pagebreak
    \chapter{Sistemas de Planejamento}
\label{ch:sistema_planejamento}
A Instalação dispõe de Sistema Computadorizado de Planejamento Tridimensional de Teleterapia:

	\noindent \textbf{Eclipse}\\
	Fabricante: Varian Medical Systems\\
	Versão: 15.6\\
	Técnicas Implementadas: 3D CRT, IMRT Sliding Window, VMAT e Elétrons.\\
\pagebreak
    \chapter{Ténicas de Tratamemento}
\label{ch:tecnicas_tratamento}
\pagebreak
    \chapter{Cálculo de Barreiras e Estimativas de Dose}
\label{ch:calculo_barreiras-estimativa_dose}
A conferência da efetividade das barreiras em manter os níveis estimados no Cálculo de Blindagens é feita periodicamente através de Levantamentos Radiométricos. Os resultados mais recentes estão apresentados na seção \ref{sec:levantamento_radiometrico}.

\pagebreak
\section{Projeto de Blindagem}
\label{sec:projeto_blindagem}
\inserirPDF{./anexos/PDFs/Cálculo de Blindagem/6EX.pdf}

\section{Levantamento Radiométrico}
\label{sec:levantamento_radiometrico}
\inserirPDF{./anexos/PDFs/Levantamento Radiométrico/Levantamento Radiometrico.pdf}
\pagebreak
    \chapter{Monitoração Individual}
\label{ch:monitoracao_individual}

\section{Normas de uso dos dosímetros}

\section{Contrato Monitoração Individual}
\label{sec:contrato_monitorcao-individual}
\inserirPDF{./anexos/PDFs/Contrato Monitoração Individual/Monitoracao Individual-CONTRATO.pdf}
\pagebreak
    \chapter{Gerência de Rejeitos Radioativos}
\label{ch:gerencia_rejeitos_radioativos}
\section{Descrição e Classificação dos Rejeitos Radioativos}


\section{Procedimentos para Coleta, Segregação, Acondicionamento e Identificação de Rejeitos Radioativos}

\section{Armazenamento em Depósito Inicial}

\begin{figure}[!h]
    \centering
    \includegraphics[width=0.75\linewidth]{./anexos/imagens/Gerencia Rejeitos Radioativos/gerencia_identificacao-balde.png}
    \caption{Balde identificado contendo cofre com fonte radioativa.}
    \label{fig:gerencia_identificacao-balde}
\end{figure}

\begin{figure}[!h]
    \centering
    \includegraphics[width=0.75\linewidth]{./anexos/imagens/Gerencia Rejeitos Radioativos/gerencia_sinalizacao-sala.png}
    \caption{Sala de tratamento trancado com chave, devidamente sinalizada e com sistema de monitoração de área.}
    \label{fig:gerencia_sinalizacao-sala}
\end{figure}

\section{Tratamento}


\section{Dispensa de Rejeitos}


\section{Registros e Inventários}

\vfill

\pagebreak
    \chapter{Descrição do Controle Médico dos IOEs - ASO}
\label{ch:descricao_controle-medico}
A tabela \ref{tab:asos}, abaixo, lista todos os IOES e a validade dos últimos atestados de saude ocupacional.

\begin{table}[!ht]
    \centering
    \resizebox{\linewidth}{!}{
    \scriptsize
    \begin{NiceTabular}{|p{8cm}|p{3.5cm}|p{3.5cm}|}\hline%
    \rowcolor{\colorTABLE}
        \color{white}\bfseries IOE & \color{white}\bfseries Último & \color{white}\bfseries Validade\\\hline\hline
        \csvreader[head to column names, late after line = \\\hline]{./anexos/CSVs/asos.csv}{}{%
        \nome & \ultimo & \validade}%
        \hline
    \end{NiceTabular}
    }
    \caption{Controle de Atestado de Saúde Ocupacional.}
    \label{tab:asos}
\end{table}

\pagebreak
    \chapter{Programa de Treinamento em Proteção Radiológica para os IOEs}
\label{ch:programa_treinamento_pr}

\vfill\pagebreak
    \chapter{Programa de Educação Continuada para os IOEs}
\label{ch:programa_educacao_continuada}

\vfill\pagebreak
    \chapter{Procedimentos de Emergência}
\label{ch:procedimentos_emergencia}

%\inserirPDF{./anexos/PDFs/Procedimentos de Emergência/Procedimentos Emergencia.pdf}

\end{anexosenv}


%%
% ESSE TEMPLATE FOI DESENVOLVIDO POR PEDRO HENRIQUE DIAS COELHO COM OBJETIVO DE MELHORAR A ARQUITETURA E CONSTRUÇÃO DE RELATÓRIO DE PROTEÇÃO RADIOLÓGICA. SEU USO É DESTINADO A SUPERVISORES DE PROTEÇÃO RADIOLÓGICA E SEUS SUBSTITUTOS. O COMPARTILHAMENTO DESSA ARQUITETURA É DE USO LIVRE, MAS O AUTOR NÃO SE RESPONSABILIZA PELO CONTEÚDO INSERIDO E NEM MESMO DAS CONSEQUENCIAS LEGAIS COMO CONSEQUENCIA DO USO INCORRETO DESSE MODELO. 

% PEDROHDCOELHO@GMAIL.COM
% (11) 9 9347 4500

%%
\end{document}