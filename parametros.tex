%%
% ESSE TEMPLATE FOI DESENVOLVIDO POR PEDRO HENRIQUE DIAS COELHO COM OBJETIVO DE MELHORAR A ARQUITETURA E CONSTRUÇÃO DE RELATÓRIO DE PROTEÇÃO RADIOLÓGICA. SEU USO É DESTINADO A SUPERVISORES DE PROTEÇÃO RADIOLÓGICA E SEUS SUBSTITUTOS. O COMPARTILHAMENTO DESSA ARQUITETURA É DE USO LIVRE, MAS O AUTOR NÃO SE RESPONSABILIZA PELO CONTEÚDO INSERIDO E NEM MESMO DAS CONSEQUENCIAS LEGAIS COMO CONSEQUENCIA DO USO INCORRETO DESSE MODELO. 

% PEDROHDCOELHO@GMAIL.COM
% (11) 9 9347 4500

%%

% --- QUANTO A NATUREZA DO DOCUMENTO
\def\titulo{Plano de Proteção Radiológica\\****}
\def\autor{Departamento de Física Médica}
\def\local{Goiânia}
\def\ano{ANO}
\def\mes{MES}
\def\instituicao{**** - ****}
\def\tipotrabalho{Plano de Proteção Radiológica}
% O preambulo deve conter o tipo do trabalho, o objetivo, 
% o nome da instituição e a área de concentração 
\def\preambulo{O Plano de Proteção Radiológica compilado documento que se segue, tem por objetivo descrever deveres e diretos de todos os Indivíduos Ocupacionalmente Expostos e também diretrizes, projetos e bem-feiturias realizados por parte do \instituicao, para que as práticas sejam realizadas em consonância com as normas da CNEN.}
% ---------------------------------------------------------------------------------------------------

% --- QUANTO A PROTECAO RADIOLOGICA
\def\titularA{TITULAR VAI AQUI}
\def\titularACPF{****}

\def\titularB{TITULAR VAI AQUI}
\def\titularBCPF{****}

\def\supervisor{SUPERVISOR VAI AQUI}
\def\supervisorRT{****}
\def\supervisorRA{****}

\def\subsSupervisor{SUBSTITUTO SUPERVISOR VAI AQUI}
\def\subsSupervisorRT{****}
\def\subsSupervisorRA{****}

\def\rt{RT VAI AQUI}
\def\rtCRM{****}
\def\rtCB{****}

\def\subsRt{SUBSTITUTO RT VAI AQUI}
\def\subsRtCRM{****}
\def\subsRtCB{****}
% ---------------------------------------------------------------------------------------------------

% --- QUANTO A INSTALACAO
\def\instalacaoNOME{NOME DA INSTALAÇÃO VAI AQUI}
\def\instalacaoMATRICULA{****}
\def\instalacaoCNPJ{****}
\def\instalacaoRUA{****}
\def\instalacaoCOMPLEMENTO{****}
\def\instalacaoBAIRRO{****}
\def\instalacaoCIDADE{****}
\def\instalacaoUF{{****}}
\def\instalacaoCEP{****}
\def\instalacaoTELEFONE{****}
\def\instalacaoGRUPO{****}
\def\instalacaoSUBGRUPO{****}
% ---------------------------------------------------------------------------------------------------

% --- QUANTO A FORMATAÇÃO E PADRÃO DE CORES
% --- mude seu padrao de cores aqui
\definecolor{color1}{RGB}{46,108,150}
\definecolor{color2}{RGB}{91, 177, 169}

% --- 
% Formato de tabelas e negrito colorido
\newcolumntype{x}[1]{>{\centering\arraybackslash\vspace{0pt}}p{#1}}
\newcolumntype{y}[1]{>{\arraybackslash\vspace{0pt}}p{#1}}
\newcommand{\COLORtextbf}[1]{\textbf{\color{color1}#1}}

\def\colorPDF{color2}
\def\colorTABLE{color2}

% --- 
\renewcommand{\thesection}{\arabic{chapter}.{\arabic{section}}}
\renewcommand{\thesection}{\arabic{chapter}.{\arabic{section}}}
\renewcommand{\familydefault}{\sfdefault}

% --- 
% Inserir PDFs
\def\escalaPDF{0.9}
\newcommand{\inserirPDF}[1]{
    \begin{tcbraster}[raster columns=1,colframe = \colorPDF,colback = white,colbacktitle = \colorPDF!100,fonttitle = \small\ttfamily,boxsep = 0pt,toptitle = 1mm,bottomtitle = 1mm,drop lifted shadow,center title,title = {\imagename\ [\imagepage]}]
        \centering \tcbincludepdf[scale=\escalaPDF , graphics orientation=portrait]{#1}
    \end{tcbraster}
}
% ---------------------------------------------------------------------------------------------------